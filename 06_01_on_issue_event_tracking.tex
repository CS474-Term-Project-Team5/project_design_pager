 In this task, we will sort events (extracted from the Second Clustering process explained above) in chronological order. First, we will categorize each article based on the time it was published. The unit of categorization, or a time frame, coule be a week or a month; it will be adjusted for better performance. Each event will appear across multiple articles. Here, we would be able to compute frequency of an event in certain time frame.\\
 For example, suppose event A (or topic A) occurs most frequently in time frame 3, and event B (or topic B) occurs most frequently in time frame 2. Then it would be reasonable to say that event B happened prior to event A (assuming time frame 3 indicates the subsequent time frame of time frame 2).\\
 There could be many different logics to infer in which time frame certain event has occurred most frequently, and the topic distribution of each article obtained in the previous task would be useful.