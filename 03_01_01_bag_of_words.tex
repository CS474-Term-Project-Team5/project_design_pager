The Bag of Words approach is a quantifying method of data by fucussing on the frequency of each words without considering the sequence of words. Generating a dictionary with tokenized Keywords and Named Entity, convert to a vector according to the frequency of each index. The pros and cons of that method are as follows.

\begin{center}
    \begin{tabularx}{\textwidth} { 
      | >{\centering\arraybackslash}X 
      | >{\centering\arraybackslash}X | }
        \hline
        Pros & Cons\\
        \hline
        {\begin{itemize}
            \item Easy to embedding
            \item Possible to embed words with Keywords and Named Entity
        \end{itemize}} &
        {\begin{itemize}
            \item Can't use similar words or Named Entity in embedding process
            \item Low weight for an important word whose frequency is small
        \end{itemize}}\\
        \hline
    \end{tabularx}
\end{center}